\documentclass[12pt]{article}

\usepackage{amsmath}
\usepackage{amsthm}
\usepackage[makeroom]{cancel}
\usepackage{amsfonts}
\usepackage{verbatim}
\usepackage{hyperref}
\usepackage[utf8]{inputenc}
\usepackage[russian]{babel}
\usepackage{graphicx}
\usepackage{alltt}
\usepackage{hyperref}

\setlength{\baselineskip}{16.0pt}

\setlength{\parskip}{3pt plus 2pt}
\setlength{\parindent}{20pt}
\setlength{\oddsidemargin}{0.5cm}
\setlength{\evensidemargin}{0.5cm}
\setlength{\marginparsep}{0.75cm}
\setlength{\marginparwidth}{2.5cm}
\setlength{\marginparpush}{1.0cm}
\setlength{\textwidth}{150mm}


\begin{document}
\begin{figure}[ht!]
\centering
\includegraphics[width=30mm]{su.jpg}
\label{overflow}
\end{figure}

\begin{center}
Софийски Университет "Св. Климент Охридски"
\end{center}

\begin{center}
\huge{Първа и втора лема на Щолц}
\end{center}

\null
\vfill
\begin{alltt}
Изготвил: Минко Гечев
Ръководител: доц. Първан Първанов
\end{alltt}

\newpage\mbox{}\newpage

\newtheorem{lemma1}{Лема}
\newtheorem*{lemma2}{Първа лема на Щолц}
\newtheorem*{lemma3}{Втора лема на Щолц}
\newtheorem{theorem}{Теорема}
\newtheorem{example}{Пример}

\section{Теореми и доказателства}
\begin{lemma1}
Нека \(\{y_n\}_{n=1}\) е строго монотонна, а
\begin{center}
\(\displaystyle\lim_{x \to \infty} \frac{x_{n+1} - x_n}{y_{n+1} - y_n} = l\) (1)
\end{center}
за \(\forall \varepsilon\ > 0, \exists N_1: \forall k > n > N_1:\)
\begin{center}
\(\displaystyle\lvert\frac{x_k - x_n}{y_k - y_n} - l\rvert < \varepsilon_1\) (2)
\newline
\end{center}
\it{Доказателство:} За да докажем лемата ще разгледаме два случая, съответно, когато редицата е строго растяща и строго намаляваща.

1 сл. Нека \({y_n}\) е строго растяща. От (1) \(\Rightarrow \exists N_1 : \forall n > N_1, n \in \mathbb N \) е изпълнено:

\begin{center}
\(\displaystyle\lvert\frac{x_{n+1} - x_n}{y_{n+1} - y_n} - l\rvert < \varepsilon_1\) (3)
\end{center}

Ще докажем, че това е търсеното \(N_1\). Нека \(n, k \in \mathbb N\) и \(k > n > N_1\).
Понеже \(y_j > y_i\) за \(\forall j > i\)

\begin{center}
\(\displaystyle\lvert\frac{x_j - x_i}{y_j - y_i} - l\rvert < \varepsilon_1 \Leftrightarrow\) 
\((l - \varepsilon_1)(y_j - y_i) < x_j - x_i < (l + \varepsilon_1)(y_j - y_i)\) (4)
\end{center}
Ако приложим (4) за \(i = n, j = n + 1\) и \(j = n + 2, i = n + 1\) и т.н. получаваме:

\begin{center}
\(
+\left\{
    \begin{array}{l l}
    (l - \varepsilon_1)(y_{n+1} - y_n) < x_{n+1} - x_n < (l + \varepsilon_1)(y_{n+1} - y_n)\\
    ...\\
    (l - \varepsilon_1)(y_k - y_{k-1}) < x_k - x_{k-1} < (l + \varepsilon_1)(y_k - y_{k-1})
    \end{array} \right.
\)
\newline
\end{center}
\begin{center}
\(
    (l - \varepsilon_1)(\cancel{y_{n+1}} - y_n + \cancel{y_{n+2}} - y_{n+1} +...+ y_k - \cancel{y_{k-1}}) <
    x_{n+1} - x_n + x_{n+2} - x_{n-1} +...+ x_k - x_{k-1} <
    (l + \varepsilon_1)(\cancel{y_{n+1}} - y_n + \cancel{y_{n+2}} - \cancel{y_{n+1}} +...+ y_k - \cancel{y_{k-1}})
\newline
\newline
    (l - \varepsilon_1)(y_k - y_n) < x_k - x_n < (l - \varepsilon_1)(y_k - y_n) (5)
\)
\end{center}
от тук намерихме \(N_1: k > n > N_1\) и
\begin{center}
    \(\displaystyle\lvert\frac{x_k - x_n}{y_k - y_n} - l\rvert < \varepsilon_1\), защото
    (5) е еквивалентно на (2).
\end{center}
Случаят за монотонно намаляваща редица се доказва аналогично.
\end{lemma1}

\begin{lemma2}
Нека \(\{y_n\}_{n=1}^\infty\) е строго монотонна.
\begin{center}
    \(\displaystyle\lim_{n \to \infty}x_n=0, \displaystyle\lim_{n \to \infty}y_n=0\)
    и
    \(\displaystyle\lim_{n \to \infty}\frac{x_{n+1} - x_n}{y_{n+1} - y_n} = l\), тогава
    \(\displaystyle\lim_{n \to \infty}\frac{x_n}{y_n} = l\)
\newline
\end{center}
\it{Доказателство:} Нека \(\varepsilon > 0, \varepsilon_1 \in (0, \varepsilon)\), то от Лема 1 \(\Rightarrow \exists N: k > n > N\)
\begin{center}
\(\displaystyle\lvert \frac{x_k - x_n}{y_k - y_n} - l\rvert < \varepsilon\)
\end{center}
Нека фиксираме \(n > N\) и направим граничен преход при \(k \to \infty\) в горното неравенство. По условие \(\{x_n\}_{n=0}^\infty\) и \(\{y_n\}_{n=0}^\infty\) са намаляващи и клонят към 0 при \(\displaystyle n \to \infty \Rightarrow \lvert\frac{x_n}{y_n} - l\rvert \leq \varepsilon_1 < \varepsilon\) т.е.:
\begin{center}
\(\displaystyle\forall n > N \Rightarrow \lvert \frac{x_n}{y_n} - l \rvert < \varepsilon \Rightarrow \lim_{n \to \infty}\frac{x_n}{y_n} = l\).
\end{center}
\end{lemma2}

\begin{lemma3}
Нека \(\{y_n\}_{n=1}^\infty\) е строго растяща - \(\displaystyle\lim_{n \to \infty}y_n = \infty\) и \(\displaystyle\lim_{n \to \infty}\frac{x_{n+1} - x_n}{y_{n+1} - y_n} = l\). Тогава:
\begin{center}
\(\displaystyle\lim_{n \to \infty}\frac{x_n}{y_n} = l\)
\newline
\end{center}
\it{Доказателство:} Нека \(\varepsilon > 0\) и \(\varepsilon_1 \in (0, \varepsilon)\). От Лема 1 следва, че \(\exists N_1 : k > m > N_1:\)
\begin{center}
\(
    \displaystyle\lvert\frac{x_k - x_m}{y_k - y_m} - l\rvert < \varepsilon_1
\)
\end{center}
т.е. \((l - \varepsilon_1)(y_k - y_m) < x_k - x_m < (l + \varepsilon_1)(y_k - y_m)\).
Нека \(m > N_1, y_m > 0\). Нека разделим на \(y_k > 0\):
\begin{center}
\(\displaystyle(l - \varepsilon_1)(l - \frac{y_m}{y_k}) < \frac{x_k}{y_k} - \frac{x_m}{y_k} < (l + \varepsilon_1)(1 - \frac{y_m}{y_k})\)
\end{center}
Като прехвърлим \(\displaystyle\frac{x_m}{y_k}\) получаваме:
\begin{center}
\(\displaystyle(l - \varepsilon_1)(1 - \frac{y_m}{y_k}) + \frac{x_m}{y_k} < \frac{x_k}{y_k} < (l + \varepsilon_1)(1 - \frac{y_m}{y_k}) + \frac{x_m}{x_k}\) (7).
\end{center}
Нека \(k \to \infty\), понеже \(y_k \to \infty\), при \(k \to \infty\):
\begin{center}
\(
    \displaystyle\lim_{k \to \infty}((l - \varepsilon_1)(1 - \frac{y_m}{y_k}) + \frac{x_m}{y_k}) = l - \varepsilon_1 < l - \varepsilon,
\)
и
\end{center}
\begin{center}
\(
    \displaystyle\lim_{k \to \infty}((l + \varepsilon_1)(1 - \frac{y_m}{y_k}) + \frac{x_m}{y_k}) = l + \varepsilon_1 < l + \varepsilon
\)
\end{center}
Следователно съществува \(N\), такова че за всяко \(k > N\) е изпълнено:
\begin{center}
\(
    \displaystyle l - \varepsilon < (l - \varepsilon_1)(1 - \frac{y_m}{y_k}) + \frac{x_m}{y_k}
\)
и
\(
    \displaystyle(l + \varepsilon_1)(1 - \frac{y_m}{y_k}) + \frac{x_m}{y_k} < l + \varepsilon 
\)
\end{center}
Като се върнем в (7) получаваме:
\begin{center}
\(l - \varepsilon < \frac{x_k}{y_k} < l + \varepsilon\) или
\(\displaystyle\forall n > N \Rightarrow \lvert \frac{x_n}{y_n} - l \rvert < \varepsilon\)
\end{center}
\end{lemma3}

\begin{theorem}

Нека \((a_n)\) е редица от реални числа, \(\displaystyle\{b_n\}_{n=0}^\infty: \lim_{n \to \infty} b_n = \infty\). Ако:
\begin{center}
\(
    \displaystyle\lim_{n \to \infty}\frac{b_n}{b_{n+1}} = b, b \in \mathbb R, b \neq 1
\)
\end{center}
То: \(\displaystyle\lim_{n \to \infty}\frac{a_n}{b_n} = l \Rightarrow \lim_{n \to \infty}\frac{a_{n+1} - a_n}{b_{n+1} - b_n} = l \)
\newline
\it{Доказателство:}
\(
    \displaystyle\frac{a_{n + 1} - a_n}{b_{n+1} - b_n} \frac{\frac{1}{b_{n+1}}}{\frac{1}{b_{n+1}}} = \frac{\frac{a_{n+1}}{b_{n+1}} - \frac{a_n}{b_n}\frac{b_n}{b_{n+1}}}{1 - \frac{b_n}{b_{n+1}}}
    \to
    \frac{l - lb}{1 - b} = l.
\)
\end{theorem}

\begin{theorem}
Нека е дадена редицата \(\{x_n\}\). Ако \(\displaystyle\lim_{n \to \infty} x_n = x, x \in (-\infty, \infty)\), тогава:
\begin{center}
\(
    \displaystyle\lim_{n \to \infty} \frac{x_1 + x_2 + ... + x_n}{n} = x
\)
\end{center}
\it{Доказателство:} Нека \(b_n = n, a_n = x_1 + x_2 + ... + x_n\), тогава \(\displaystyle\frac{a_{n+1} - a_n}{b_{n+1} - b_n} = x_{n+1} \to x\).
\end{theorem}

\newpage
\section{Примери}

\begin{example}
Оценете \(\displaystyle\lim_{n \to \infty} \frac{1^k + 2^k + ... + n^k}{n^{k+1}}\), където \(k \in \mathbb N\).
\newline
\it{Решение:} Нека \(a_n = 1^k + 2^k + ... + n^k, b_n = n^{k+1}\). Ясно е, че редицата \(b_n\) е с положителни членове, строго растяща и неограничена. Сега:
\begin{center}
\(
    \displaystyle\lim_{n->\infty}\frac{a_{n+} - a_n}{b_{n+1} - b_n} = \lim_{n \to \infty}\frac{(n + 1)^k}{(n + 1)^{k+1} - n^{k+1}} = 
    \lim_{n \to \infty}\frac{(n + 1)^k}{(1 + \binom{k+1}{1}n + \binom{k + 1}{2}n^2+ ... + \binom{k + 1}{k}n^k + n^{k + 1}) - n^{k + 1}} = 
    \lim_{n \to \infty}\frac{(n + 1)^k/n^k}{(1 + \binom{k + 1}{1}n + \binom{k + 1}{2}n^2 + ... + \binom{k + 1}{k}n^k)/n^k} =
    \lim_{n \to \infty}\frac{(1 + 1 / n)^k}{\binom{k + 1}{k}} = \frac{1}{k + 1}.
\)
\end{center}
От тук използвайки лемата на Щолц получаваме, че границата е \(\frac{1}{k + 1}\).
\end{example}

\begin{example}
Оценете \(\displaystyle\lim_{n \to \infty}\frac{\sum_{k = 1}^nka_n}{n^2}\), при \(\displaystyle\lim_{n \to \infty}a_n = L\).
\newline
\it{Решение:} От втората лема на Щолц следва, че горната редица има същата граница като:
\begin{center}
\(\displaystyle\frac{\sum_{k = 1}^n ka_k - \sum_{k = 1}^{n - 1} ka_k}{n^2 - (n - 1)^2} = \frac{na_n}{2n - 1} = \frac{a_n}{2 - 1/n} \to \frac{L}{2}\)
\end{center}
\end{example}

\begin{example}
Нека \(\{x_n\}\) е редица от реални числа и нека:
\(x_{n + 1} = x_n + e^{-x_n}, \forall n \geq 0.\)
\newline
Оценете: \(\displaystyle\lim_{n\to\infty} (x_n - \ln(n+1))\).
\newline
\it{Решение:} Тъй като \(x_n\) е растяща и \(x_n \to \infty\).
\newlineОзначаваме \(\displaystyle y_n=e^{x_n-\ln(n+1)}=\frac{e^{x_n}}{n+1}\).
Нека приложим лемата на Щолц:
\newline
\(\displaystyle\frac{e^{x_{n+1}}-e^{x_n}}{n+2-(n+1)}=e^{x_n}(e^{x_{n+1}-x_n}-1)=\frac{e^{e^{-x_n}}-1}{e^{-x_n}}\to 1\)
\newline
Тъй като \(\displaystyle \lim_{y\to 0}\frac{e^y-1}{y}=1\) и \(e^{-x_n}\to 0\).
\newline
От Щолц следва, че \(y_n \to 1\), което означава, че \(\displaystyle\lim_{n \to \infty}x_n-\ln(n+1)= 0\).
\end{example}

\begin{example}
Намерете:
\begin{center}
\(\displaystyle\lim_{n\to\infty}\frac{1!+2!+\cdots+n!}{n!}\)
\end{center}
\it{Решение:}
Като приложим лемата на Щолц получаваме:
\begin{center}
\(\displaystyle\lim_{n\to\infty}\frac{1!+2!+\cdots+n!}{n!}=\lim_{n\to\infty}\frac{(n+1)!}{(n+1)!-n!}=\lim_{n\to\infty}\frac{n+1}{n}=1\)
\end{center}
\end{example}

\begin{example}
Дадена е редицата: \(\displaystyle a_1 = 1, a_{n+1} = a_n + \frac{1}{a_n}\), намерете
\(\displaystyle\lim_{n \to \infty}\frac{a_n}{n}\).
\it{Решение:}
Прилагайки лемата на Щолц получаваме:
\begin{center}
\(\displaystyle\lim_{n\to \infty} \frac{a_n}{n}=\lim_{n\to\infty} \frac{a_{n+1}-a_{n}}{(n+1)-n}=\lim_{n\to \infty}\frac{\frac{1}{a_{n}}}{1}=\lim_{n\to \infty}\frac{1}{a_{n}}=0\).
\end{center}
\end{example}

\newpage
\bf{Литература:}
\begin{enumerate}
    \item Уикипедия - \href{https://en.wikipedia.org/wiki/Stolz\%E2\%80\%93Ces\%C3\%A0ro_theorem}{en.wikipedia.org/wiki/Stolz theorem}
    \item Marian Mureşan: A Concrete Approach to Classical Analysis. Springer 2008
\end{enumerate}
\end{document}
